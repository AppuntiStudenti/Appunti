\documentclass[11pt]{book}
\usepackage{hyperref}
\usepackage{colonequals}
\usepackage{amsfonts}
\usepackage{amssymb}
\usepackage{amsmath}
\usepackage[utf8]{inputenc}
\usepackage[T1]{fontenc}
\usepackage{float}
\usepackage{fixltx2e}
\usepackage[italian]{babel}
\usepackage{graphicx}

\newenvironment{sistema}%
{\left\lbrace\begin{array}{@{}l@{}}}%
{\end{array}\right.}


\title{Appunti di Ricerca Operativa}
\author{Fabio Viola}
\date{}

\setcounter{chapter}{8}

\begin{document}

\chapter{Rilassamenti}

\scriptsize
{\bf Slide}:
\href{http://www.or.deis.unibo.it/staff_pages/martello/Chapter9.zip}{Relaxations}
\normalsize
\vspace{20pt}

Gli algoritmi branch and bound e branch and cut sono gli algoritmi
pi\`u utilizzati per la soluzione ottima di problemi
$\mathcal{NP}$-difficili \footnote{Un problema H \`e
  $\mathcal{NP}$-difficile se e solo se esiste un problema
  $\mathcal{NP}$-completo L che \`e polinomialmente riducibile ad H,
  ovvero tale che $L \leq_T H$. In altre parole, L deve poter essere
  risolto in tempo polinomiale da una macchina di Turing dotata di un
  oracolo per H. Da questa definizione si ricava che i problemi
  $\mathcal{NP}$-difficili sono non meno difficili dei problemi
  $\mathcal{NP}$-completi, che a loro volta sono per definizione i più
  difficili delle classi P/NP.}.


Per far si che questi algoritmi siano efficaci sono necessari alcuni
ingredienti:
\begin{itemize}
\item la progettazione dell'albero decisionale e della strategia di
  eslporazione;
\item l'uso di rilassamenti per calcolare bound e per realizzare
  procedure di riduzione;
\item l'uso di algoritmi approssimati per determinare rapidamente
  soluzioni ammissibili.
\end{itemize}

\section{Generalit\`a}

Considereremo un problema di massimizzazione $\mathcal{NP}$-difficile definito da
un insieme F delle soluzioni ammissibili (supposto finito) e da una
funzione obiettivo $z:F \rightarrow R^1$ che associa ad un punto della
regione ammissibile un valore. La coppia $(z,F)$ indicher\`a il
problema di individuare l'elemento $y^* \in F$ (cio\`e il punto $y^*$
della regione ammissibile) tale che:

\begin{center}
$z(y^*) \geq z(y) \forall y \in F$
\end{center}

cio\`e tale che la funzione $z$ calcolata in quel punto sia maggiore
che in ogni altro punto.


Riprendiamo il branch and bound. Sia $P^0 = (z, F(P^0))$ il problema
da risolvere ($F(P^0)$ \`e chiaramente la regione ammissibile del
problema originale $P^0$). Sia $Z(P^0) = \max \{ z(y) : y \in F(P^0)
\}$ il valore della soluzione ottima del problema $P^0$. Il metodo
branch and bound consiste nella suddivisione del problema originale
$P^0$ in sottoproblemi $P^1,\dots,P^{n_0}$ tali per cui la loro unione
rappresenti il problema originale. La suddivisione si ottiene
dividendo la regione ammissibile $F(P^0)$ del problema originale in
sottoinsiemi $F(P^1),\dots,F(P^{n_0})$ tali che la loro unione sia
pari a $F(P^0)$. Se per un generico sottoproblema $P^k$ la soluzione
\`e $Z(P^k) = \max \{ Z(P^1), \dots, Z(P^{n_0})\}$, allora la soluzione
del problema originale sar\`a data dal massimo delle soluzioni dei
sottoproblemi (trattandosi di un problema di massimo), dunque $Z(P^0)
= \max \{ Z(P^1), \dots, Z(P^{n_0}) \}$. In pratica risolvere il
problema $P^0$ corrisponde a risolvere $P^k$ con $k = 1, \dots,
n_0$. In un algoritimo branch and bound, risolvere pu\`o avere tre
diversi significati:

\begin{itemize}
\item determinare la soluzione ottima di $P^k$;
\item dimostrare che la regione ammissibile del problema $P^k$ \`e vuota;
\item dimostrare che la soluzione del problema $P^k$ \`e minore o
  uguale di $Z$ che \`e il valore della miglior soluzione ammissibile
  ottenuta in precedenza.
\end{itemize}

Generalmente \`e preferibile fare in modo che le regioni ammissibili
dei sottoproblemi non si sovrappongano, per limitare l'ampiezza dei
problemi. (In formule: $F(P^i) \cup F(P^j) = \emptyset \forall P^i,
P^j : i \neq j$).

Per ogni problema che non si riesce a risolvere (dove con risolvere
intendiamo il concetto espresso sopra) si ripete la suddivisione in
sottoproblemi ed in questo modo viene dunque generato l'albero
decisionale che \`e una caratteristica del branch and bound).

Aspetto fondamentale degli algoritmi di branch and bound \`e il calcolo
per ogni sottoproblema $P^k$ di un valore $U(P^k)$, che chiamiamo
upper bound, tale che $U(P^k) \geq Z(P^k)$, cio\`e tale che la
soluzione del problema $P^k$ sia minore o uguale all'upper bound. In
questo modo \`e possibile risolvere il sottoproblema come nel terzo
punto del precedente elenco, cio\`e possiamo verificare che l'upper
bound sia o meno minore o uguale ala miglior soluzione ammissibile
ottenuta fino a quel momento. Come facciamo a trovare un upper bound
(o un lower bound se il problema \`e di minimizzazione)? Si fa tramite
i rilassamenti.

Un rilassamento di un problema $P^k = (z, F(P^k))$ \`e un problema
$R(P^k) = (z_r, F_r(P^k))$ che soddisfa due condizioni:

\begin{enumerate}
\item $F(P^k) \subseteq F_r(P^k)$, cio\`e la regione ammissibile del
  sottoproblema $P^k$ \`e contenuta nella regione ammissibile del
  rilassamento;
\item $z_r(y) \geq z(y) \forall y \in F(P^k)$, cio\`e la funzione
  rilassamento, per ogni suo punto deve assumere un valore maggiore o
  uguale a quello del sottoproblema $P^k$.
\end{enumerate}

L'upper bound sar\`a quindi definito come:
\begin{center}
$U(P^k) = Z(R(P^k)) = \max \{ z_r(y) : y \in F_r (P^k) \}$
\end{center}

La definizione che abbiamo dato ora di rilassamento \`e pi\`u ampia di
quella utilizzata per il rilassamento continuo per il quale si ha
solamente $F(P^k) \subset F_r(P^k)$. Inoltre con la definizione di
rilassamento data, possiamo anche modificare la funzione obiettivo in
modo tale da soddisfare la condizione 2. Come si realizza il
rilassamento? Vi sono a disposizione pi\`u modi differenti, che seguono
due criteri spesso contrastanti:

\begin{itemize}
\item il valore della soluzione del rilassamento dev'essere il pi\`u
  basso possibile in modo da essere il pi\`u possibile prossimo al
  valore della soluzione ottima del sottoproblema $P^k$;
\item il problema rilassato dev'essere semplice da calcolare.
\end{itemize}

I tipi di rilassamento che presenteremo saranno illustrati facendo
riferimento al problema LP01:

\vspace{11pt}
\begin{center}
\begin{tabular}{l}
$Z(P) = \max \sum \limits_{j=1}^n v_j x_j$\\
$\phantom{Z(P) = max} \sum \limits_{j=1}^n a_{ij}x_j \leq b_i \qquad (i=1,\dots,m)$\\
$\phantom{Z(P) = max} \sum \limits_{j=1}^n d_{kj}x_j = e_k \qquad (k=1,\dots,l)$\\
$\phantom{Z(P) = max} x_j \in \{0,1\} \qquad (j=1,\dots,n)$
\end{tabular}
\end{center}
\vspace{11pt}


Chiaramente i vincoli potrebbero essere tutti espressi come
disequazioni o come equazioni, ma manterremo la distinzione per
evidenziare propriet\`a diverse. I rilassamenti che vedremo si
estendono ad un ILP generale.

\section{Rilassamento per eliminazione}

Il metodo pi\`u semplice per rilassare un problema \`e quello di
eliminare uno o pi\`u vincoli in modo da ottenere un problema ben
strutturato (TODO: cosa intende con ben strutturato?) e quindi pi\`u
semplice da risolvere. Eliminando i vincoli si ottiene lo stesso
problema, ma con una regione ammissibile pi\`u ampia.

Supponiamo per esempio di voler eliminare dal problema LP01 indicato
in precedenza tutti i vincoli con equazioni. Otterremmo:

\vspace{11pt}
\begin{center}
\begin{tabular}{l}
$Z(E(P)) = \max \sum \limits_{j=1}^n v_j x_j$\\
$\phantom{Z(E(P)) = max} \sum \limits_{j=1}^n a_{ij}x_j \leq b_i \qquad (i=1,\dots,m)$\\
$\phantom{Z(E(P)) = max} x_j \in \{0,1\} \qquad (j=1,\dots,n)$
\end{tabular}
\end{center}
\vspace{11pt}

Questo problema \`e noto come problema knapsack multidimensionale 0-1
che \`e un caso particolare di problema knapsack. In questo tipo di
problema ogni oggetto ha $m$ pesi, infatti abbiamo $m$
sommatorie/vincoli. Questi che abbiamo volgarmente chiamato pesi,
possono essere ad esempio un peso vero e proprio, un volume, un indice
di tossicit\`a ecc. Se eliminassimo anche tutti i vincoli con
disequazioni lasciandone solo uno, otterremmo un problema KP01.

\`E necessaria un'osservazione: se la soluzione ottima $x^*$ di $E(P)$
\`e ammissibile per il problema originale $P$ (cio\`e rispetta comunque
i vincoli eliminati), allora chiaramente \`e ottima anche per $P$.

\section{Rilassamento continuo}

Sempre con riferimento al problema LP01, il rilassamento continuo
consiste nel sostituire i vincoli $x_j \in \{0,1\}$ con $0 \leq x_j
\leq 1$. In questo modo si ottiene un problema di programmazione
lineare che chiamiamo $C(P)$. Se i coefficienti della funzione
obiettivo sono interi, l'upper bound pu\`o essere rinforzato come
$\lfloor Z(C(P))\lfloor$ (TODO: perche e cosa intende?). Anche in
questo caso la soluzione ottima $x^*$ di $C(P)$ \`e ammissibile per $P$
(cio\`e \`e intera), allora \`e ottima per $P$.

Possiamo osservare che per un generico rilassamento $R$:
\begin{itemize}
\item se il rilassamento $R(P)$ \`e impossibile, allora lo \`e anche $P$
\item se $R(P)$ \`e illimitato, non si pu\`o dire nulla su $P$. Nel caso
  del rilassamento continuo, se $C(P)$ \`e illimitato, allora $P$ pu\`o
  solo essere illimitato o impossibile.
\end{itemize}

\section{Rilassamento surrogato}

Vediamo in due passi come si realizza questo rilassamento:

\begin{enumerate}

\item Si sceglie un insieme di vincoli (ad esempio le disequazioni
  del solito problema LP01) e si aggiunge al problema originale {\em
    P} il vincolo ridondante:

  \begin{center}
    $\sum\limits_{i=1}^m \pi_i \sum\limits_{j=1}^n a_{ij}x_j \leq
    \sum_{i=1}^m \pi_i b_i$
  \end{center}

  con $\pi_i \geq 0 \forall i$ {\bf moltiplicatori surrogati}. In
  pratica abbiamo moltiplicato sia a destra che a sinistra per la
  sommatoria di $\pi_i$ che va da {\em 1} a {\em m}. Il problema \`e
  inalterato perch\'e ogni soluzione ammissibile per il nuovo problema
  \`e anche ammissibile per il problema originale.

\item Si eliminano i vincoli scelti (cio\`e le disequazioni
  precedenti) ottenendo il {\bf rilassamento surrogato}:

  
  \vspace{11pt}
  \begin{center}
    \begin{tabular}{l}
      $Z(S(P,\pi)) = \max \sum \limits_{j=1}^n v_j x_j$\\
      $\phantom{Z(S(P,\pi)) = max} \sum \limits_{j=1}^n \hat{a}_{ij}x_j \leq \hat{b}$\\
      $\phantom{Z(S(P,\pi)) = max} \sum \limits_{j=1}^n d_{kj}x_j = e_k \qquad (k=1,\dots,l)$\\
      $\phantom{Z(S(P,\pi)) = max} x_j \in \{0,1\} \qquad (j=1,\dots,n)$
    \end{tabular}
  \end{center}
  \vspace{11pt}    
  
  dove $\hat{a}_j = \sum\limits_{i=1}^m \pi_ia_{ij}$ ($j=1,\dots,n$) e
  $\hat{b} = \sum\limits_{i=1}^m \pi_ib_i$. In poche parole abbiamo
  sostituito un insieme di vincoli con il vincolo (chiaramente pi\`u
  debole) ottenuto da una loro somma pesata.  

\end{enumerate}

Anche qui, se la soluzione ottima $x^*$ di $S(P,\pi)$ \`e ammissibile
per $P$ allora \`e la soluzione ottima per $P$. Il valore di
$Z(S(P,\pi))$ \`e un upper bound valido per qualunque vettore di
moltiplicatori surrogati non negativi. A questo punto \`e giusto
dunque chiedersi {\bf come} si trova il miglior vettore di
moltiplicatori surrogati, cio\`e quello che produce l'upper bound
pi\`u basso. Questo problema \`e detto {\bf problema surrogato duale};
consiste nel determinare:

\begin{center}
$Z(S(P,\pi^*)) = \min\limits_{\pi \geq 0}\{ Z(S(P,\pi))\}$  
\end{center}

\section{Rilassamento lagrangiano}

Mentre i rilassamenti visti finora utilizzano soltanto la condizione
A, cio\`e sostituiscono la regione ammissibile con una pi\`u ampia
senza variare la funzione obiettivo, i rilassamenti lagrangiani
utilizzano anche la condizione B, dunque oltre ad ampliare la regione
ammissibile modificano anche la funzione obiettivo.

Dal momento che tale rilassamento cambia a seconda che si applichi ad
equazioni o disequazioni, vedremo separatamente i due casi, iniziando
dal secondo.

\subsection{Applicazione del rilassamento lagrangiano a disequazioni}

Procediamo illustrando i due passi che ci portano ad ottenere il
rilassamento:

\begin{enumerate}

\item Si sceglie un insieme di disequazioni (ad esempio tutte le
  disequazioni) e si aggiunge alla funzione obiettivo un termine che
  sia non negativo in tutti i punti che rispettano i vincoli (cio\`e
  un termine che sia non negativo nella regione ammissibile).

  Scelti {\em m} {\bf moltiplicatori lagrangiani} $\lambda_i \geq 0$,
  la funzione obiettivo viene modificata in modo da penalizzare
  l'eventuale violazione dei vincoli:

  \begin{center}
    $\max \sum\limits_{j=1}^n v_jx_j + \sum\limits_{i=1}^m
    \lambda_i(b_i - \sum\limits_{j=1}^n a_{ij}x_j)$
  \end{center}

  Il problema cos\`i riformulato fornisce quindi un upper bound
  valido. 

%% Questa formulazione pu\`o essere ricondotta con un fantasioso esempio
%% ad un cercatore di funghi che \`e obbligato a restare in una
%% determinata zona, ma pagando una penale pu\`o avventurarsi anche nella
%% zona proibita. La penale cresce con l'allontanarsi dal confine. Se il
%% cercatore rimane invece nella zona

\item Si eliminano i vincoli scelti, ottenendo il {\bf rilassamento
  lagrangiano}: 

\vspace{11pt}
\begin{center}
\begin{tabular}{l}
$Z(L(P,\lambda)) = \sum\limits_{i=1}^m \lambda_i b_i +  \max \sum \limits_{j=1}^n \hat{v}_j x_j$\\
$\phantom{Z(P) = max} \sum \limits_{j=1}^n d_{kj}x_j = e_k \qquad (k=1,\dots,l)$\\
$\phantom{Z(P) = max} x_j \in \{0,1\} \qquad (j=1,\dots,n)$
\end{tabular}
\end{center}
\vspace{11pt}

dove $\hat{v}_j = v_j - \sum\limits_{i=1}^m \lambda_i a_{ij}$, ($j =
1, \dots, n$). Si osservi che il termine $\sum\limits_{i=1}^m
\lambda_i b_i$ \`e indipendente da $x$.
  
\end{enumerate}

Rispetto agli altri rilassamenti abbiamo una differenza importante: in
questo, una soluzione ottima $x^*$ per il rilassamento $L(P,\lambda)$
non \`e detto che sia ottima per $P$, perch\'e, lo ricordiamo, in
questo tipo di rilassamento abbiamo facolt\`a di variare la funzione
di partenza. L'ottimo coincide se sia {\em P} che $L(P, \lambda)$
hanno lo stesso valore nel punto ottimo. Questo caso si verifica se:

\begin{center}
$\sum\limits_{i=1}^m \lambda_i (b_i - \sum\limits_{j=1}^n a_{ij}x_j^*)
  = 0$
\end{center}

cio\`e se la soluzione del rilassamento ha moltiplicatore nullo per
ogni vincolo o se il vincolo \`e soddisfatto con uguaglianza per ogni
moltiplicatore non nullo. Questa condizione \`e sufficiente, ma non
necessaria. 

Il {\bf problema lagrangiano duale} \`e determinare $\lambda^*$ tale
che:

\begin{center}
$Z(L(P,\lambda^*)) = \min_{\lambda \geq 0} \{ Z(L(P,\lambda)) \}$  
\end{center}

\subsection{Applicazione del rilassamento lagrangiano ad equazioni}


\begin{enumerate}
  
\item Scegliamo in questo caso un insieme di equazioni, ad esempio
  tutte quelle del nostro problema LP01. A questo punto aggiungiamo
  alla funzione obiettivo un termine che, al contrario del caso con le
  disequazioni, sia nullo in ogni punto che rispetta i vincoli (cio\`e
  in ogni punto della regione ammissibile). Scelti $l$ {\bf
    moltiplicatori lagrangiani} $\lambda_k \gtreqless 0$, la funzione
  modificata \`e:

  \begin{center}
    $\max \sum\limits_{j=1}^n v_jx_j + \sum\limits_{k=1}^l \lambda_k
    (e_k - \sum\limits_{j=1}^n d_{kj}x_j)$
  \end{center}

  In questo caso vanno bene anche moltiplicatori negativi.

\item Adesso eliminiamo le equazioni scelte ed otteniamo il {\bf
  rilassamento lagrangiano}:

  \vspace{11pt}
  \begin{center}
    \begin{tabular}{l}
      $Z(L(P,\lambda)) = \sum\limits_{k=1}^l \lambda_ke_k + \max \sum
      \limits_{j=1}^n \hat{v}_j x_j$\\

      $\phantom{Z(Spi)) = max} \sum\limits_{j=1}^n a_{ij}x_j \leq
      b_i \qquad (i=1,\dots,m)$\\ 
      $\phantom{Z(Spi)) = max} x_j \in \{0,1\} \qquad (j=1,\dots,n)$\\
    \end{tabular}
  \end{center}
  \vspace{11pt}    

  dove $\hat{v}_j = v_j - \sum\limits_{k=1}^l \lambda_k d_{kj}$. Se la
  soluzione ottima $x^*$ di $L(P,\lambda)$ \`e ammissibile per {\em
    P}, allora essa \`e ottima per {\em P}. Infatti in qualunque
  soluzione ammissibile il termine aggiuntivo della funzione obiettivo
  \`e nullo, per cui pu\`o capitare solo che la soluzione ottima del
  rilassamento sia fuori dalla regione ammissibile di {\em P}, o che
  se \`e all'interno della regione ammissibile, allora coincide con la
  soluzione ottima di {\em P}. 

  Il {\bf problema lagrangiano duale} \`e:

  \begin{center}
  $Z(L(P, \lambda^*)) = \min\limits_\lambda \{ Z(L(P, \lambda))\}$
  \end{center}


\end{enumerate}


\section{Rilassamento per decomposizione}

Quando la struttura del problema lo consente (TODO: e cio\`e quando?)
pu\`o convenire definire due rilassamenti pi\`u semplici, la cui
somma dia un upper bound valido per il problema originale.

\begin{enumerate}

\item si sceglie un sottoinsieme di vincoli (ad esempio le equazioni
  del nostro esempio LP01) e lo si esprime mediante nuove variabili
  {\em y}. Si sceglie una parte della funzione obiettivo e si esprime
  anch'essa in funzione di {\em y}. Questa e l'altra parte della
  funzione obiettivo andranno pesate con due moltiplicatori $\alpha$ e
  $\beta$ tali che $\alpha + \beta = 1$ (spesso si scelgono entrambi
  pari a $\frac{1}{2}$). Se a questo punto si aggiunge il vincolo che
  $x$ ed $y$ abbiano lo stesso valore (TODO: e che senso ha???) si
  ottiene un problema equivalente a quello dato:

  \vspace{11pt}
  \begin{center}
    \begin{tabular}{l}
      $\max \alpha \sum \limits_{j=1}^n v_j x_j + \beta
      \sum\limits_{j=1}^n v_jy_j$\\
      $\phantom{= max} \sum \limits_{j=1}^n a_{ij}x_j \leq b_i \qquad (i=1,\dots,m)$\\
      $\phantom{= max} \sum \limits_{j=1}^n d_{kj}x_j = e_k \qquad (k=1,\dots,l)$\\
      $\phantom{= max} x_j \in \{0,1\} \qquad (j=1,\dots,n)$\\
      $\phantom{= max} y_j \in \{0,1\} \qquad (j=1,\dots,n)$\\
      $\phantom{= max} y_j = x_j \qquad (j=1,\dots,n)$\\
    \end{tabular}
  \end{center}
  \vspace{11pt}    

\item Si esegue il rilassamento lagrangiano dei nuovi vincoli
  ottenendo il {\bf rilassamento per decomposizione}:

  \vspace{11pt}
  \begin{center}
    \begin{tabular}{l}
      $Z(D(P,\lambda)) = \max \sum \limits_{j=1}^n (\alpha v_j + \lambda_j) x_j +
      \sum\limits_{j=1}^n (\beta v_j - \lambda_j)y_j$\\
      $(A)\phantom{= max} \sum \limits_{j=1}^n a_{ij}x_j \leq b_i \qquad (i=1,\dots,m)$\\
      $(B)\phantom{= max} \sum \limits_{j=1}^n d_{kj}x_j = e_k \qquad (k=1,\dots,l)$\\
      $(C)\phantom{= max} x_j \in \{0,1\} \qquad (j=1,\dots,n)$\\
      $(D)\phantom{= max} y_j \in \{0,1\} \qquad (j=1,\dots,n)$\\
    \end{tabular}
  \end{center}
  \vspace{11pt}    
  
\end{enumerate}

Sia $DX(P,\lambda)$ il problema nelle variabili {\em x} di
massimizzare il primo termine del rilassamento sotto i vincoli A e
C. Sia $DY(P,\lambda)$ il problema nelle variabili {\em y} di
massimizzare il secondo termine della funzione obiettivo sotto i
vincoli B e D. SI ha allora l'upper bound $Z(D(P, \lambda)) = Z(DX(P,
\lambda)) + Z(DY(P, \lambda))$.

Trattandosi di un rilassamento lagrangiano di equazioni, se le
equazioni ottime $x^*$ e $y^*$ dei due problemi sono uguali, allora
$x^*$ \`e ottima per {\em P}. Il {\bf problema di decomposizione
  duale} \`e:

\begin{center}
$Z(D(P, \lambda^*)) = \min_\lambda\{ Z(D(P,\lambda)) \}$  
\end{center}


\end{document}

\documentclass[11pt]{book}
\usepackage[usenames,dvipsnames,svgnames,table]{xcolor}

\parindent=0pt \bigskipamount=20pt

\begin{document}

\chapter*{Esercizio 3}

Da un deposito si devono servire $m$ clienti. Tali clienti possono
essere serviti tramite $n$ viaggi diversi. Ciascun cliente $i$
($i=1,\dots,m$) pu\`o essere servito da un sottoinsieme $v_i$ di
viaggi (con $V_i \subset \{1,\dots,n\}$). Ciascun viaggio $j$
($j=1,\dots,n$) ha un costo $c_j$ ed un tempo di viaggio $t_j$ (con
$c_j$ e $t_j$ non negativi).

Si deve determinare un sottoinsieme $S$ degli $n$ viaggi in modo che:

\begin{itemize}
\item ciascun cliente sia servito da almeno un viaggio di $S$;
\item la somma dei costi dei viaggi di $S$ sia minima;
\item la somma dei tempi di viaggio dei viaggi di $S$ sia non
  inferiore ad un valore prefissato $d$.
\end{itemize}

Un possibile modello \`e:

\begin{center}
\begin{tabular}{lp{2cm}ll}
$z = \min \sum\limits_{j=1}^n c_jx_j$ & & & (a)\\
$\qquad \sum\limits_{j\in V_i}x_j \geq 1$ & & $i=1,\dots,m$ & (b)\\
$\qquad \sum\limits_{j=1}^n t_jx_j \geq d$ & & & (c)\\
$\qquad x_j \in\{0,1\}$ & & $j = 1,\dots,n$ & (d) \\
\end{tabular}
\end{center}

\section*{Punto 1}

\textit{Determinare \textbf{buoni} lower bound di tipo lagrangiano
  aventi complessit\`a $O(r+n)$ (con $r=\sum_{i=1}^m|V_i|$) e
  descrivere le corrispondenti procedure subgradiente.}

\subsection*{Rilassamento lagrangiano 1}

Il vincolo (b) veniva fornito anche nella forma:

$$
\sum\limits_{j=1}^n a_{ij}x_j \geq 1 \qquad i=1,\dots,m
$$

ma noi preferiamo nel modo indicato precedentemente in quanto solo per
esaminare la matrice $A_{ij}$ abbiamo una complessit\`a di $O(n\cdot
m)$, quindi sfo\-re\-rem\-mo il vincolo dato che $r = \sum_{i=1}^m |V_i|$
\`e sempre $\leq m\cdot n$. Per definire l'insieme dei $V_i$ abbiamo una
complessit\`a $O(r)$. 

\

Il modello \`e molto simile a quello del \textbf{Set Covering
  Problem}; l'unica differenza risiede nella presenza del vincolo
$(c)$. Un possibile rilassamento \`e quindi quello che coinvolge
questo vincolo. Portandolo in funzione obiettivo otterremmo
l'SCP. Questo problema \`e $\mathcal{NP}$-difficile, perci\`o va
rilassato ulteriormente se vogliamo risolverlo in tempo proporzionale
a $r$.

Per risolvere SCP dobbiamo quindi fare un altro rilassamento che pu\`o
ad esempio essere il \textbf{rilassamento lagrangiano} dei vincoli di
copertura (b), oppure il \textbf{rilassamento surrogato} degli stessi
(che porta a KP-min il cui rilassamento continuo si risolve in
$O(r)$). Tutti i rilassamenti che abbiamo visto per SCP portano a
problemi che si risolvono in tempo proporzionale al numero di elementi
pari a 1 nella matrice $A_{ij}$, quindi $r$.

\

Quando abbiamo fatto il rilassamento del SCP abbiamo definito oltre
agli insiemi $J_i$ (che corrispondono ai nostri $V_i$) anche gli
insiemi $I_j$. I primi sono gli insiemi contenenti le colonne $j$ che
coprono la riga $i$, i secondi gli insiemi contenenti le righe $i$
coperte dalla colonna $j$. Qui li chiamiamo $B_j$
($j=1,\dots,n$). Questi sottoinsiemi ci servono per poter calcolare in
modo efficiente i costi lagrangiani. Vediamo come si possano ricavare
i $B_j$ dai $V_i$:

$$
B_j := \{ i : j\in V_i\}
$$

La procedura per costruirli \`e dunque:

\vspace{20pt}
\begin{tabular}{l}
\textbf{FOR} $j=1,\dots,n$ \textbf{DO} $B_j := 0$\\
\textbf{FOR} $i=1,\dots,m$ \\
$\qquad$ \textbf{FOR} $j \in V_i$ \textbf{DO}\\
$\qquad\qquad B_j := B_j \cup \{i\}$\\
\end{tabular}
\vspace{20pt}

Abbiamo ottenuto in tempo $O(n)$ per l'inizializzazione pi\`u $O(r)$
per riempire i sottoinsiemi, quindi $O(r+n)$. Stiamo ancora dentro il
numero di esecuzioni.

\

Ritorniamo dunque ai rilassamenti. Facciamo il rilassamento
lagrangiano dei vincoli di copertura (b). Abbiamo bisogno di $m$
moltiplicatori $\lambda_i \geq 0$. I vincoli da rilassare verranno
inseriti in funzione obiettivo nella forma \textit{secondo membro meno
  primo membro}. Usiamo questa forma per ottenere un termine $\leq 0$
e ci\`o \`e necessario dato che il problema \`e di minimizzazione.

\begin{center}
\begin{tabular}{lp{2cm}ll}
$z(\lambda) = \min \sum\limits_{j=1}^n c_jx_j - \sum\limits_{i=1}^m
\lambda_i(1-\sum\limits_{j\in V_i}x_j)$ & & & (a)\\
$\qquad \sum\limits_{j=1}^n t_jx_j \geq d$ & & & (c)\\
$\qquad x_j \in\{0,1\}$ & & $j = 1,\dots,n$ & (d) \\
\end{tabular}
\end{center}

Fatto ci\`o possiamo definire i costi lagrangiani. Riscriviamo la
funzione o\-biet\-ti\-vo

$$
z(\lambda) = \sum\limits_{i=1}^m \lambda_i + \min \sum\limits_{j=1}^n
c_jx_j + \sum\limits_{i=1}^m\sum\limits_{j\in V_i}\lambda_ix_j
$$

Adesso facciamo un passaggio che ci consente di portare anche
nell'ultima parte dell'espressione una sommatoria con $j$ che va da
$1$ a $n$ in modo da poter raccogliere le $x_j$. Per farlo teniamo a
mente la seguente equivalenza:

$$
\sum\limits_{i=1}^m\sum\limits_{j\in V_i} =
\sum\limits_{j=1}^n\sum\limits_{i\in B_j}
$$

da cui:

$$
z(\lambda) = \sum\limits_{i=1}^m \lambda_i + \min \sum\limits_{j=1}^n
c_jx_j + \sum\limits_{j=1}^n\sum\limits_{i\in B_j}\lambda_ix_j
$$

Come detto raccogliamo le $x_j$:

$$
z(\lambda) = \sum\limits_{i=1}^m \lambda_i + \min \sum\limits_{j=1}^n
(c_j + \sum\limits_{i \in B_j}\lambda_i)x_j
$$

Quindi i costi lagrangiani saranno definiti da:

$$
\tilde{c}_j := c_j + \sum\limits_{i \in B_j}\lambda_i \qquad \forall
j=1,\dots,n
$$

Determinare i costi lagrangiani in questo modo ha complessit\`a
$O(r)$, infatti il calcolo si svolge in questo modo: 

\vspace{20pt}
\begin{tabular}{l}
\textbf{FOR} $j=1,\dots,n$ \textbf{DO}\\
$\qquad \tilde{c}_j := c_j$\\
$\qquad$\textbf{FOR} $i \in B_j$ \textbf{DO}\\
$\qquad\qquad \tilde{c}_j := \tilde{c}_j + \lambda_i$
\end{tabular}
\vspace{20pt}

La funzione obiettivo quindi \`e (tralasciando la sommatoria
iniziale):

\begin{center}
\begin{tabular}{lp{2cm}ll}
$z(\lambda) = \min \sum\limits_{j=1}^n \tilde{c}_jx_j$ & & & (a)\\
$\qquad \sum\limits_{j=1}^n t_jx_j \geq d$ & & & (c)\\
$\qquad x_j \in\{0,1\}$ & & $j = 1,\dots,n$ & (d) \\
\end{tabular}
\end{center}

Quello ottenuto \`e un problema KP01-min che ha per\`o una
particolarit\`a perch\'e i costi $\tilde{c}_j$ possono anche essere
$\leq 0$. Non possiamo perci\`o applicare Balas-Zemel. Cosa facciamo?
Prendiamo sicuramente tutte le variabili con costo lagrangiano
negativo perch\'e fanno decrescere la funzione obiettivo e ci fanno
soddisfare pi\`u facilmente il vincolo (c). Facciamo quindi un
preprocessing che controlla il segno e assegna $x_j=1$ alle variabili
con costo negativo. Se dopo il loro inserimento il vincolo (c) non \`e
ancora soddisfatto, ci resta un problema in cui rimangono solo le
variabili con costo strettamente positivo e in cui la capacit\`a non
\`e $d$, ma quella residua dopo aver preso le variabili negative,
$\bar{d}$. A questo problema ridotto possiamo applicare
Balas-Zemel. La soluzione si trova quindi in $O(n)$.

\

La procedura subgradiente \`e esattamente analoga a quella vista a
lezione per SCP.

\subsection*{Rilassamento lagrangiano 2}

L'altro rilassamento possibile \`e quello che porta il vincolo (b) in
funzione obiettivo. Non essendoci pi\`u il vincolo di copertura avremo
KP01-min multiplo che \`e un problema
$\mathcal{NP}$-difficile. Rilassando i vincoli di copertura in maniera
surrogata otteniamo KP01-min. Facendone il rilassamento continuo si
pu\`o applicare Balas-Zemel possiamo risolverlo in tempo $O(n)$. Il
calcolo dei costi lagrangiani si effettua in $O(r)$ come segue:

\vspace{20pt}
\begin{tabular}{l}
\textbf{FOR} $j=1,\dots,n$ \textbf{DO}\\
$\qquad w_j := 0$\\
\textbf{FOR} $j=1,\dots,n$ \textbf{DO}\\
$\qquad$ \textbf{FOR} $i\in B_j$ \textbf{DO}\\
$\qquad\qquad w_j := w_j + u_i$
\end{tabular}
\vspace{20pt}

\section*{Punto 2}

Determinare un buon lower bound di tipo surrogato avente complessit\`a
$O(r+n)$ e descrivere la corrispondente procedura subgradiente.

\

Facciamo il rilassamento lagrangiano del vincolo (c) che essendo un
unico vincolo richieder\`a un unico moltiplicatore $\lambda \geq
0$. Il vincolo verr\`a portato in f.o. nella forma \textit{secondo
  membro meno primo membro}.

\begin{center}
\begin{tabular}{lp{2cm}ll}
  $z(\lambda) = \min \sum\limits_{j=1}^n c_jx_j + \lambda (d -
  \sum\limits_{j=1}^n t_jx_j))$ & & & (a)\\
  $\qquad \sum\limits_{j\in V_i}x_j \geq 1$ & & $i=1,\dots,m$ & (b)\\
  $\qquad x_j \in\{0,1\}$ & & $j = 1,\dots,n$ & (d) \\
\end{tabular}
\end{center}

quindi con ovvi passaggi:

\begin{center}
\begin{tabular}{lp{2cm}ll}
  $z(\lambda) = \lambda d + \min \sum\limits_{j=1}^n (c_j-\lambda t_j)x_j$ & & & (a)\\
  $\qquad \sum\limits_{j\in V_i}x_j \geq 1$ & & $i=1,\dots,m$ & (b)\\
  $\qquad x_j \in\{0,1\}$ & & $j = 1,\dots,n$ & (d) \\
\end{tabular}
\end{center}

da cui ricaviamo che i costi lagrangiani sono $\tilde{c}_j = c_j -
\lambda t_j$ (definiti in $O(n)$) ed il modello risultante \`e:

\begin{center}
\begin{tabular}{lp{2cm}ll}
  $z(\lambda) = \lambda d + \min \sum\limits_{j=1}^n \tilde{c}_jx_j$ & & & (a)\\
  $\qquad \sum\limits_{j\in V_i}x_j \geq 1$ & & $i=1,\dots,m$ & (b)\\
  $\qquad x_j \in\{0,1\}$ & & $j = 1,\dots,n$ & (d) \\
\end{tabular}
\end{center}

\`e un problema SCP, ma con costi che possono essere negativi a
differenza dell'SCP originale. I rilassamenti che abbiamo visto per il
set covering si applicano a istanze in cui tutte le colonne hanno
costi sono $\geq 0$. Cosa facciamo allora? Intanto vediamo che il
problema \`e di minimizzazione, quindi se le colonne hanno costi
negativi le inseriamo senz'altro in soluzione in $O(n)$. Fatto ci\`o
bisogna quindi vedere se tutti i vincoli di copertura sono gi\`a
soddisfatti. Non \`e detto che tutti lo siano, quindi potrebbe
rimanere un problema SCP ridotto da risolvere. In questo problema le
colonne son solo quelle con costi maggiori di 0, mentre le righe son
quelle non ancora co\-per\-te. Il problema ha $O(n)$ colonne e $O(r)$
elementi pari a 1 nella matrice dei coefficienti. Ovviamente il numero
di colonne e di righe \`e minore o uguale a quello delle colonne e
righe di partenza. Abbiamo ottenuto un problema
$\mathcal{NP}$-difficile che possiamo rilassare con uno qualunque dei
rilassamenti visti per risolverlo in $O(r)$. Posso fare quindi il
rilassamento lagrangiano dei vincoli rimasti, oppure il rilassamento
surrogato dei vincoli di copertura (che richiede $O(r)$ per definire i
costi surrogati) e che ci porta ad un problema KP01 il cui
rilassamento pu\`o essere risolto con Balas-Zemel.

\

La procedura subgradiente non \`e diversa da quella utilizzata nel
problema 1.

\section*{Punto 3}

Determinare un rilassamento surrogato che fornisca un buon lower bound
in $O(r+n)$ e descrivere la procedura subgradiente.

\

Abbiamo vari possibili rilassamenti surrogati. Cerchiamo di avere un
unico vincolo di copertura per ricondurci a KP01-min, problema di cui
poi realiz\-ze\-rem\-mo il rilassamento continuo.

\

Rilassiamo dunque tutti i vincoli (b) ed il vincolo (c). Associamo ad
ogni vincolo di tipo (b) un moltiplicatore $\lambda_i\geq 0$ ed al
vincolo (c) un moltiplicatore $\sigma \geq 0$. I vincoli sono entrambi
nella forma $\geq$, pertanto, dopo averli moltiplicati per i relativi
moltiplicatori, possono essere sommati.

\begin{center}
\begin{tabular}{lp{2cm}ll}
$z(\lambda,\sigma) = \min \sum\limits_{j=1}^n c_jx_j$ & & & (a)\\
$\qquad \sum\limits_{i=1}^m \lambda_i\sum\limits_{j\in V_i}x_j +
\sigma\sum\limits_{j=1}^nt_jx_j \geq \sum\limits_{i=1}^n \lambda_i +
\sigma d$\\
$\qquad x_j \in\{0,1\}$ & & $j = 1,\dots,n$ & (d) \\
\end{tabular}
\end{center}


Cerchiamo di raccogliere i coefficienti di $x_j$. Facciamo perci\`o lo
stesso passaggio visto prima:

$$
\sum\limits_{j=1}^n \sum\limits_{i\in B_j} \lambda_i x_j +
\sigma\sum\limits_{j=1}^nt_jx_j \geq \sum\limits_{i=1}^n \lambda_i +
\sigma d
$$

Quindi:

$$
\sum\limits_{j=1}^n (\sum\limits_{i\in B_j} \lambda_i +
\sigma t_j)x_j \geq \sum\limits_{i=1}^n \lambda_i +
\sigma d
$$

che riscriviamo pi\`u comodamente come:

$$
\sum\limits_{j=1}^n \tilde{p}_jx_j \geq \tilde{d}
$$

I profitti surrogati sono stati definiti in tempo $O(r)$ perch\'e il
calcolo di tutti i $\tilde{p}_j$ richiede in fondo di fare tante somme
quanti sono gli elementi degli insiemi $B_j$ che in tutto sono $r$.

\

Il problema ottenuto \`e:

\begin{center}
\begin{tabular}{lp{2cm}ll}
$z(\lambda,\sigma) = \min \sum\limits_{j=1}^n c_jx_j$ & & & (a)\\
$\qquad \sum\limits_{j=1}^n \tilde{p}_jx_j \geq \tilde{d}$ & & & (bc) \\
$\qquad x_j \in\{0,1\}$ & & $j = 1,\dots,n$ & (d) \\
\end{tabular}
\end{center}

\`E un problema KP01 in forma di minimo. Tutti i $\tilde{p}_j$ ed i
costi sono $\geq 0$ quindi \`e proprio un classico KP01. Possiamo
farne il rilassamento continuo e risolviamo con Balas-Zemel in tempo
$O(n)$.

\

La procedura subgradiente \`e quella vista per la soluzione del
problema 1.

\end{document}
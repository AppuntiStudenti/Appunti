\documentclass[11pt]{book}
\usepackage[usenames,dvipsnames,svgnames,table]{xcolor}
\usepackage[color=yellow]{todonotes}
\hyphenation{pro-fit-ti}

\parindent=0pt \bigskipamount=20pt

\begin{document}

\chapter*{Esercizio 8}

Sia dato un grafo orientato completo $G=(V,A)$ con
$V=\{1,2,\dots,n\}$. Ad ogni arco $(i,j) \in A$ sono associati un peso
$p_{ij}$ ed un tempo $t_{ij}$ (con $t_{ij}\geq 0$). Siano inoltre dati
due sottoinsiemi disgiunti di archi $S$ e $T$ (con $S \subset A$,
$T\subset A$, $S\cap T = \emptyset$).Si deve costruire un circuito
hamiltoniano H tale che:

\begin{itemize}
\item la somma dei pesi degli archi di H sia massima;
\item la somma dei tempi degli archi di H sia non superiore ad un
  valore dato $d$
\item il numero di archi di H appartenenti a S sia non inferiore al
  numero di archi di H appartenenti a T
\item per il rilassamento della terza domanda si definisca la
  procedura subgradiente con complessit\`a non superiore ad $O(n^3)$
\end{itemize}

Un possibile modello \`e:

\begin{center}
\begin{tabular}{lp{2cm}ll}
  $z = \max \sum\limits_{i=1}^n\sum\limits_{j=1}^n p_{ij}x_{ij}$ & & & (1)\\
  $\qquad \sum\limits_{j=1}^n x_{ij} = 1$ & & $i=1,\dots,n$ & (2)\\
  $\qquad \sum\limits_{i=1}^n x_{ij} = 1$ & & $j=1,\dots,n$ & (3)\\
  $\qquad \sum\limits_{i \in R}\sum\limits_{j \in V \textbackslash R}
  x_{ij} \geq 1$ & & $\forall R \subset V : r \in R$ & (4) \\
  $\qquad \sum\limits_{i=1}^n\sum\limits_{j=1}^n t_{ij}x_{ij} \leq d$ &
  & & (5) - b \\
  $\qquad \sum\limits_{(i,j)\in S} x_{ij} \geq \sum\limits_{(i,j) \in
    T}x_{ij}$ &&& (6) - c \\
  $\qquad x_j \in\{0,1\}$ & & $j,i = 1,\dots,n$ & (7) \\
\end{tabular}
\end{center}

dove $r$ \`e un vertice qualsiasi di $V$.

\section*{Punto 1}

\textit{Si determini un rilassamento lagrangiano relativo ai vincoli b e c
risolubile in $O(n^3)$.}

\

Trattandosi di vincoli di disequazione i moltiplicatori saranno $\geq
0$. Il pro\-ble\-ma \`e di massimo, quindi i termini da aggiungere
alla f.o. devono essere $\geq 0$, pertanto i vincoli \textit{c}
verranno posti nella forma \textit{primo membro meno secondo membro},
mentre i vincoli \textit{b} nella forma \textit{secondo membro meno
  primo membro}. La funzione obiettivo si pu\`o scrivere come:

$$
\max \sum\limits_{i=1}^n\sum\limits_{j=1}^np_{ij}x_{ij} + \lambda
\biggr (d
-\sum\limits_{i=1}^n\sum\limits_{j=1}^n t_{ij}x_{ij}\biggr ) + \sigma
\biggr (\sum\limits_{(i,j) \in S}x_{ij} - \sum\limits_{(i,j) \in
  T}x_{ij} \biggr )
$$

Esplicitando i vari termini:

$$
\lambda d + \max \sum\limits_{i=1}^n\sum\limits_{j=1}^n
\biggr (p_{ij}-\lambda t_{ij} \biggr )x_{ij} + \sigma
\biggr (\sum\limits_{(i,j) \in S}x_{ij} - \sum\limits_{(i,j) \in
  T}x_{ij} \biggr )
$$

A questo punto vorremmo definire i pro\-fit\-ti lagrangiani, ma ci
troviamo con due sommatorie piuttosto fastidiose. La soluzione
consiste nel definire i profitti lagrangiani in maniera differenziata
a seconda che l'arco $(i,j)$ appartenga a S, a T oppure a nessuno dei
due:

\begin{itemize}
\item se $(i,j) \in S$: $\tilde{p}_{ij} := p_{ij}-\lambda t_{ij} + \sigma$
\item se $(i,j) \in T$: $\tilde{p}_{ij} := p_{ij}-\lambda t_{ij} - \sigma$
\item altrimenti:  $\tilde{p}_{ij} := p_{ij}-\lambda t_{ij}$
\end{itemize}

La definizione dei profitti lagrangiani richiede $O(n^2)$. Fatto
ci\`o, il problema rimanente \`e:

\begin{center}
\begin{tabular}{lp{2cm}ll}
  $z(\lambda) = \lambda d + \max \sum\limits_{i=1}^n\sum\limits_{j=1}^n \tilde{p}_{ij}x_{ij}$ & & & (1)\\
  $\qquad \sum\limits_{j=1}^n x_{ij} = 1$ & & $i=1,\dots,n$ & (2)\\
  $\qquad \sum\limits_{i=1}^n x_{ij} = 1$ & & $j=1,\dots,n$ & (3)\\
  $\qquad \sum\limits_{i \in R}\sum\limits_{j \in V \textbackslash R}
  x_{ij} \geq 1$ & & $\forall R \subset V : r \in R$ & (4) \\
  $\qquad x_j \in\{0,1\}$ & & $j,i = 1,\dots,n$ & (7) \\
\end{tabular}
\end{center}

Ci\`o che abbiamo ottenuto \`e esattamente il problema ATSP di
massimo. Non c'\`e alcun bisogno di passare alla sua formulazione di
minimo! Sappiamo che applicando l'\textbf{AP relaxation} otteniamo un
problema rilassato che pu\`o essere risolto in $O(n^3)$. Eliminiamo
quindi i vincoli di connessione.

\section*{Punto 2}

\textit{Per il rilassamento della domanda precedente si definisca la
  corrispondente procedura subgradiente con complessit\`a $O(n^3)$.}

\

Come valori per inizializzare la procedura possiamo prendere $UB:=\infty$,
$\lambda:=0$ e $\sigma :=0$. Ad ogni iterazione si deve risolvere il
rilassamento (AP relaxation) con i moltiplicatori $\lambda$ e $\sigma$
correnti ed eventualmente aggiornare l'upper bound con $UB := \min
\{UB, z(\lambda, \sigma) \}$.

Poi bisogna aggiornare i moltiplicatori e per fare ci\`o abbiamo
bisogno di definire i subgradienti:

\begin{itemize}
\item $s(\lambda) := d - \sum\limits_{i=1}^n\sum\limits_{j=1}^n
  t_{ij}x_{ij}$
\item $s(\sigma) := \sum\limits_{(i,j)\in S}x_{ij} -
  \sum\limits_{(i,j) \in T}x_{ij}$
\end{itemize}

Entrambi i vincoli \textit{b} e \textit{c} sono violati se i
subgradienti sono minori di 0, pertanto le procedure di aggiornamento
dei moltiplicatori sono:

\begin{itemize}
\item $\lambda := \max \{ 0, \lambda - \beta s(\lambda)\}$
\item $\sigma := \max \{0, \sigma - \beta s(\sigma)\}$
\end{itemize}

Notiamo che le procedure di aggiornamento dei moltiplicatori cercano
sempre il massimo, anche se il problema di partenza \`e di
minimizzazione.

\

Quante iterazioni posso fare della procedura subgradiente? Ci viene
chiesto di non sforare $O(n^3)$ pertanto il numero di iterazioni
dovr\`a essere rappresentato da una costante e non da un numero
dipendente da $n$.

\section*{Punto 3}

\textit{Determinare un ulteriore rilassamento lagrangiano dei vincoli b e c
(ed eventualmente altri vincoli) che fornisca un buon lower bound in
$O(n^2)$.
}
\

Procediamo esattamente come al punto 1 per definire il rilassamento
lagrangiano dei vincoli \textit{b} e \textit{c}. Ci\`o che cambia \`e
che non effettuiamo l'AP relaxation, ma un \textbf{rilassamento
  lagrangiano dei vincoli di out degree}. Dal momento che dopo il
primo rilassamento lagrangiano abbiamo trovato un ATSP di massimo, se
ora rimuoviamo i vincoli di out degree, non ci troviamo davanti ad un
problema che pu\'o essere scomposto in Shortest Spanning Arborescence
e Minimum Cost Arc Entering $r$, ma abbiamo \textbf{Longest Spanning
  Arborescence rooted in $r$} e \textbf{Maximum Cost Arc Entering
  $r$}. Il problema di trovare l'arborescenza di costo massimo pu\`o
essere risolto in $O(n^2)$, mentre quello di trovare l'arco entrante
in $r$ di costo massimo in $O(n)$, quindi la complessit\`a del calcolo
del lower bound \`e $O(n^2)$ come richiesto.

\section*{Punto 4}

\textit{Per il rilassamento della domanda precedente si definisca la
  corrispondente procedura subgradiente con complessit\`a $O(n^3)$.}

\

Il procedimento \`e analogo a quello visto al punto 2 con la
differenza che il metodo per la risoluzione del rilassamento ha
complessit\`a $O(n^2)$ e non $O(n^3)$, perci\`o, dato che ci viene
chiesta una procedura con complessit\`a $O(n^3)$ possiamo effettuare
anche un numero di iterazioni dipendente da $n$.

\end{document}